% !TEX root = ../main.tex
\newpage
\ssection{Заключение}

При выполнении работы было спроектировано и реализовано программное средство,
позволяющее пользователям работать с зашифрованными текстовыми файлами.

К достоинствам разработанной программы можно отнести:
\begin{itemize}
  \item Простоту пользовательского интерфейса;
  \item Достаточный функционал;
  \item Наличие русского и английского переводов.
\end{itemize}

К недостаткам относятся:
\begin{itemize}
  \item Снижение скорости шифрования из-за частичного использования
  Python в модулях шифрования;
  \item Использование метода шифрования EBC;
\end{itemize}

Основываясь на результатах работы можно сделать следующие выводы:
\begin{itemize}
  \item Использование ассемблерного языка замедляет процесс написания
  программного кода и усложняет отладку программы.
  \item При использовании языка C упрощается процесс разработки и отладки
  программы в сравнении с языком ассемблера. Таким образом, язык C удобен
  для написания участков кода, от которых требуется высокая
  скорость выполнения.
  \item Использование интерпретируемого языка Python целесообразно для
  написания участков кода, от которых не требуется высокая скорость выполнения.
  Python значительно упрощает разработку и отладку программы в сравнении
  с языком C.
\end{itemize}
