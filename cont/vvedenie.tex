% !TEX root = ../main.tex
\newpage
\ssection{Введение}

На современном этапе развития IT-технологий имеются большие возможности
для несанкционированного доступа к хранимой информации. Существенную роль
играет вопрос о защите данных пользователя.

Для защиты данных от несанкционированного доступа используется множество
криптографических методов:
\begin{itemize}
    \item Симметричное шифрование;
    \item Асимметричное шифрование;
    \item Цифровые подписи;
    \item Хеширование.
\end{itemize}

Существует большое количество средств, использующих криптографические методы для
защиты данных пользователя, шифрования сообщений, подписания документов и т.д.

Сегодня у человека существует потребность в хранении большого объема конфиденциальной
информации. К этому понятию можно отнести персональные данные, телефонные номера,
контакты, пароли, личные заметки. Для решения этой задачи разрабатываются
пакеты прикладных программ, облачные сервисы и клиентские приложения.

Целью работы является проектирование и разработка программного средства
с графическим интерфейсом для создания, редактирования и просмотра
зашифрованных текстовых документов ``на лету''.

Для достижения этой цели были поставлены следующие задачи:
\begin{enumerate}
    \item Провести исследование и составить характеристику существующих
    пакетов и клиентских приложений, предоставляющих аналогичный функционал;
    \item На основании проведенного исследования и обзора определить
    задачи по разработке программного средства.
\end{enumerate}
