% !TEX root = ../main.tex
\newpage
\renewcommand{\refname}{Список источников} % \bibname
%\bibliographystyle{gost780s} % ГОСТ 7.80
\begin{thebibliography}{9}
\addcontentsline{toc}{section}{\refname}

\bibitem{gnupg}
  Лозовюк А. GnuPG -- OpenSource шифрование и цифровые подписи.
  [Электронный ресурс]. -- Режим доступа: \\
  \verb'http://citforum.ru/security/cryptography/gnupg/', свободный.

\bibitem{python-docs}
  Документация по Python 3.4.3 [Электронный ресурс], 1990-2015. -- Режим доступа: \\
  \verb'https://docs.python.org/3.4/', свободный.

\bibitem{qt5-docs}
  Документация по Qt 5.5 [Электронный ресурс], 2015. -- Режим доступа: \\
  \verb'http://doc.qt.io/qt-5/', свободный.

\bibitem{panasenko}
  Панасенко С. П. Алгоритмы шифрования. Специальный справочник. --
  СПб.: БХВ-Петербург, 2009. -- 576 с.: ил.

\bibitem{feal-attack}
  Matsui M., Yamagishi A. A New Method for Known Plaintext Attack of
  FEAL Cipher. [Электронный ресурс], 1992. -- Режим доступа: \\
  \verb'http://link.springer.com/content/pdf/10.1007%2F3-540-47555-9_7.pdf'

\bibitem{blowfish}
  Schneier, B. Description of a New Variable-Length Key, 64-Bit Block Cipher (Blowfish)
  [Электронный ресурс]. -- Режим доступа: \\
  \verb'https://www.schneier.com/cryptography/archives/1994/09/description_of_a_new.html',
  свободный.

\end{thebibliography}


% FEAL-NX Specs:
%     http://info.isl.ntt.co.jp/crypt/eng/archive/dl/feal/call-3e.pdf
% Blowfish Specs:
%
% Blowfish encryption algorithm:
%     Bruce Schneier
%     https://www.schneier.com/cryptography/archives/1994/09/description_of_a_new.html
%
%
% Обзор GPG:
%     http://citforum.ru/security/cryptography/gnupg/
